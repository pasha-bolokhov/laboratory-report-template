%%%%%%%%%%%%%%%%%%%%%%%%%%%%%%%%%%%%%%%%%%%%%%%%%%%%%%%%%%%%%%%%%%%%%%%%%%%%%%%%
%%%%%%%%%%%%%%%%%%%%%%%%%%%%%%%%%%%%%%%%%%%%%%%%%%%%%%%%%%%%%%%%%%%%%%%%%%%%%%%%
%                                                                              %
%                            D O C U M E N T   C A P                           %
%                                                                              %
%%%%%%%%%%%%%%%%%%%%%%%%%%%%%%%%%%%%%%%%%%%%%%%%%%%%%%%%%%%%%%%%%%%%%%%%%%%%%%%%
%%%%%%%%%%%%%%%%%%%%%%%%%%%%%%%%%%%%%%%%%%%%%%%%%%%%%%%%%%%%%%%%%%%%%%%%%%%%%%%%
\documentclass[epsfig,12pt]{article}
\usepackage{epsfig}
\usepackage{graphicx}
\usepackage{rotating}
\usepackage{latexsym}
\usepackage{amsmath}
\usepackage{amssymb}
\usepackage{relsize}
\usepackage[top=0.5in, right=0.5in, left=0.5in, bottom=0.8in]{geometry}
\geometry{letterpaper}
\usepackage{color}
\usepackage[usenames,dvipsnames,svgnames,table]{xcolor}
\usepackage{bm}
\usepackage{slashed}
\usepackage{enumitem}
\usepackage{tabularx}
\usepackage{booktabs}
\usepackage{multirow}
\usepackage{wrapfig}
\usepackage{fixmath}
\usepackage{esint}
\usepackage{verbatim}
\usepackage{wesa}
\usepackage{aurical}
\usepackage{pifont}
\usepackage[colorlinks]{hyperref}
\usepackage{array}
\usepackage{pgfornament}
\usepackage{siunitx}
\usepackage[europeanresistors]{circuitikz}
\usetikzlibrary{decorations.pathmorphing,patterns,arrows.meta,calc,decorations.markings}
\usetikzlibrary{math,shadows,3d,shapes,decorations,intersections}




%%%%%%%%%%%%%%%%%%%%%%%%%%%%%%%%%%%%%%%%%%%%%%%%%%%%%%%%%%%%%%%%%%%%%%%%%%%%%%%%
%%%%%%%%%%%%%%%%%%%%%%%%%%%%%%%%%%%%%%%%%%%%%%%%%%%%%%%%%%%%%%%%%%%%%%%%%%%%%%%%
%                                                                              %
%                       D O C U M E N T   S E T T I N G S                      %
%                                                                              %
%%%%%%%%%%%%%%%%%%%%%%%%%%%%%%%%%%%%%%%%%%%%%%%%%%%%%%%%%%%%%%%%%%%%%%%%%%%%%%%%
%%%%%%%%%%%%%%%%%%%%%%%%%%%%%%%%%%%%%%%%%%%%%%%%%%%%%%%%%%%%%%%%%%%%%%%%%%%%%%%%
\def\baselinestretch{1.1}
% Make the bullet list symbol a small dot
\renewcommand{\labelitemi}{{\Large$\cdot$}}
%
% This setting adjusts the units to be displayed with a central dot symbol for multiplication, e.g.
%     \si{20}{\kg\metre\per\square\second} 
% will display as
%     20 kg·m/s²
\sisetup{per-mode = symbol,
	inter-unit-product = \ensuremath{{}\cdot{}}}



%%%%%%%%%%%%%%%%%%%%%%%%%%%%%%%%%%%%%%%%%%%%%%%%%%%%%%%%%%%%%%%%%%%%%%%%%%%%%%%%
%%%%%%%%%%%%%%%%%%%%%%%%%%%%%%%%%%%%%%%%%%%%%%%%%%%%%%%%%%%%%%%%%%%%%%%%%%%%%%%%
%                                                                              %
%                                D O C U M E N T                               %
%                                                                              %
%%%%%%%%%%%%%%%%%%%%%%%%%%%%%%%%%%%%%%%%%%%%%%%%%%%%%%%%%%%%%%%%%%%%%%%%%%%%%%%%
%%%%%%%%%%%%%%%%%%%%%%%%%%%%%%%%%%%%%%%%%%%%%%%%%%%%%%%%%%%%%%%%%%%%%%%%%%%%%%%%
\begin{document}


%%%%%%%%%%%%%%%%%%%%%%%%%%%%%%%%%%%%%%%%%%%%%%%%%%%%%%%%%%%%%%%%%%%%%%%%%%%%%%%%
%                                                                              %
%                                                                              %
%                                   T I T L E                                  %
%                                                                              %
%                                                                              %
%%%%%%%%%%%%%%%%%%%%%%%%%%%%%%%%%%%%%%%%%%%%%%%%%%%%%%%%%%%%%%%%%%%%%%%%%%%%%%%%
\begin{center}
	\textbf{ \huge \color{RoyalBlue} Report for Laboratory \underline{\hskip 1cm}  }\\[2mm]
	\textbf{\textit{ \Large \color{VioletRed} Title of the Laboratory  }}\\[2mm]
	\textit{ \large \color{BlueViolet} Your Course Name  }
\end{center}
\hrule
\bigskip
\begin{tabularx}{\textwidth}{lXr}
%
	&
	\textbf{Name:}
	&
	\textbf{Laboratory Section:}
	\hspace{1.0cm}
	\\[2mm]
%
	&
	\textbf{Partners:}
\end{tabularx}
\medskip


%%%%%%%%%%%%%%%%%%%%%%%%%%%%%%%%%%%%%%%%%%%%%%%%%%%%%%%%%%%%%%%%%%%%%%%%%%%%%%%%
%                                                                              %
%                                 P U R P O S E                                %
%                                                                              %
%%%%%%%%%%%%%%%%%%%%%%%%%%%%%%%%%%%%%%%%%%%%%%%%%%%%%%%%%%%%%%%%%%%%%%%%%%%%%%%%
\section*{\textit{Purpose}}

	The purpose in no more than one sentence --- typically a laboratory tests a law or a principle,
	usually in the form of an equation (for example $ F ~=~ m a $).
	More rarely the goal of the laboratory is the acquaintance with 
	certain concepts (such as energy, momentum, force, voltage \emph{etc})


%%%%%%%%%%%%%%%%%%%%%%%%%%%%%%%%%%%%%%%%%%%%%%%%%%%%%%%%%%%%%%%%%%%%%%%%%%%%%%%%
%                                                                              %
%                               P R O C E D U R E                              %
%                                                                              %
%%%%%%%%%%%%%%%%%%%%%%%%%%%%%%%%%%%%%%%%%%%%%%%%%%%%%%%%%%%%%%%%%%%%%%%%%%%%%%%%
\section*{\textit{Procedure}}

	The most difficult part --- the core of understanding the laboratory.
	In at most four sentences show how the goal is achieved.
	For example, if the laboratory tests the law $ F ~=~ m a $,
	you briefly describe how you measure (or calculate) the force,
	and how you measure the acceleration.
	That is, you are independently experimentally measuring the left-- and right--hand sides 
	of the equation, and then checking that they agree with each other.
	The \emph{results} are not discussed here.
	Use present tense for your discussion
	

%%%%%%%%%%%%%%%%%%%%%%%%%%%%%%%%%%%%%%%%%%%%%%%%%%%%%%%%%%%%%%%%%%%%%%%%%%%%%%%%
%                                                                              %
%                            O B S E R V A T I O N S                           %
%                                                                              %
%%%%%%%%%%%%%%%%%%%%%%%%%%%%%%%%%%%%%%%%%%%%%%%%%%%%%%%%%%%%%%%%%%%%%%%%%%%%%%%%
\section*{\textit{Observations}}

	Here go the measurements.
	Data is most efficiently represented in the form of tables.
	However, before (or sometimes after) each table you need to explain what you were measuring.
	Each \emph{quantity} must be assigned a \emph{unique} letter, and must be accompanied by
	the appropriate units.
	If there is a group of similar measurements, it is recommended to specify the units
	only once per group, like in the table below.
	Each \emph{letter} must come defined, \emph{i.e.} do not assume that $ F $ is force or $ P $ is pressure,
	unless you specifically say so.
	Deep technical details are not necessary,
	but it must still be clear how each number was measured
	
\bigskip\noindent
	Example of a data table (which must be preceded by explanations),
\begin{center}
\begin{tabular}{ccc}
%
\toprule
%
	Trial	&	$ F $ (\si{\newton})	&	$ a $ (\si{\metre\per\square\second})	\\[2mm]
%
\midrule
%
	1	&	7.88	&	3.80	\\[2mm]
	2	&	7.95	&	4.08	\\[2mm]
	3	&	8.20	&	4.14	\\[2mm]
	4	&	8.12	&	3.92	\\[2mm]
%
\bottomrule
\end{tabular}
\end{center}
	For the above measurements, we used the hanging mass $ m ~=~ \SI{2.0}{\kg} $.
	The heavier mass of the counter--weight was $ M ~=~ 5.0 \pm \SI{0.01}{\kg} $.


%%%%%%%%%%%%%%%%%%%%%%%%%%%%%%%%%%%%%%%%%%%%%%%%%%%%%%%%%%%%%%%%%%%%%%%%%%%%%%%%
%                                                                              %
%                            C A L C U L A T I O N S                           %
%                                                                              %
%%%%%%%%%%%%%%%%%%%%%%%%%%%%%%%%%%%%%%%%%%%%%%%%%%%%%%%%%%%%%%%%%%%%%%%%%%%%%%%%
\section*{\textit{Calculations}}

	This is the Calculations or Analysis section.
	Short calculations can go into the Observations section, if it makes sense.
	For instance, the calculation of the average speed or mass can be quoted
	in the Observations section.
	You \emph{do not need} to explain how to calculate the average.
	However, more involved calculations need an explanation.
	A formula must be quoted whenever applicable

\bigskip\noindent
	We calculate the energy of Millenium Falcon according to the Einstein--Palpatine formula,
\[
	E  ~~=~~  \gamma\, m c^2,
\]
	where $ m $ is the mass of the spacecraft, $ c $ is the speed of light
	and $ \gamma $ is the relativistic factor.
	Using the values $ m ~=~ \SI{45000}{\kg} $, $ c ~=~ \SI{3.0d8}{\metre\per\second} $
	and $ \gamma ~=~ 1.4 $ we find
\[
 	E ~=~ \SI{5.7d21}{\joule}.
\]
	We expect therefore, the gravitational warp drive to be the most effective
	when the spacecraft has already exited into the hyperspace

\bigskip\noindent
	Repetitive calculations must be gathered into a result table,
\begin{center}
\begin{tabular}{ccc}
%
\toprule
%
	Spacecraft	&	$ m $ ($\times 10^3\,\si{\kg}$)	&	$ R $ (kiloparsec)	\\[2mm]
%
\midrule
%
	Millenium Falcon	&	45	&	1.6	\\[2mm]
	Radiant VII		&	620	&	2.1	\\[2mm]
	Rogue Shadow		&	58	&	1.7	\\[2mm]
	CR90 Corvette		&	440	&	2.4	\\[2mm]
%
\bottomrule
\end{tabular}
\end{center}
	Here $ m $ is the gross relativistic mass of the spacecraft, and $ R $ is the optimal shot range

\bigskip\noindent
	Error analysis also belongs here.
	However, if this analysis is quite short, it can be moved to Conclusion section
	if this makes more sense


%%%%%%%%%%%%%%%%%%%%%%%%%%%%%%%%%%%%%%%%%%%%%%%%%%%%%%%%%%%%%%%%%%%%%%%%%%%%%%%%
%                                                                              %
%                              C O N C L U S I O N                             %
%                                                                              %
%%%%%%%%%%%%%%%%%%%%%%%%%%%%%%%%%%%%%%%%%%%%%%%%%%%%%%%%%%%%%%%%%%%%%%%%%%%%%%%%
\section*{\textit{Conclusion}}


\newpage
%%%%%%%%%%%%%%%%%%%%%%%%%%%%%%%%%%%%%%%%%%%%%%%%%%%%%%%%%%%%%%%%%%%%%%%%%%%%%%%%
%                                                                              %
%                   L A B O R A T O R Y   E V A L U A T I O N                  %
%                                                                              %
%%%%%%%%%%%%%%%%%%%%%%%%%%%%%%%%%%%%%%%%%%%%%%%%%%%%%%%%%%%%%%%%%%%%%%%%%%%%%%%%
\section*{\textit{Laboratory evaluation}}

	Please provide feedback on the following areas, comparing this laboratory to your previous labs.
	Please assign each of the listed categories a value in 1 ~--~ 5, with 5 being the best, 1 the worst.

\bigskip
\begin{itemize}
\item
	how much fun you had completing this laboratory?
	\hfill 1\qquad 2\qquad 3\qquad 4\qquad 5

\item
	how well the lab preparation period explained this laboratory?
	\hfill 1\qquad 2\qquad 3\qquad 4\qquad 5

\item
	the amount of work required compared to the time allotted?
	\hfill 1\qquad 2\qquad 3\qquad 4\qquad 5

\item
	your understanding of this laboratory?
	\hfill 1\qquad 2\qquad 3\qquad 4\qquad 5

\item
	the difficulty of this laboratory?
	\hfill 1\qquad 2\qquad 3\qquad 4\qquad 5

\item
	how well this laboratory tied in with the lecture?
	\hfill 1\qquad 2\qquad 3\qquad 4\qquad 5
\end{itemize}

\bigskip\noindent
	Comments supporting or elaborating on your assessment can also be very helpful in improving the future laboratories
	--- replace this text with your comments


\end{document}
