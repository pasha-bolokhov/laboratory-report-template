%%%%%%%%%%%%%%%%%%%%%%%%%%%%%%%%%%%%%%%%%%%%%%%%%%%%%%%%%%%%%%%%%%%%%%%%%%%%%%%%
%%%%%%%%%%%%%%%%%%%%%%%%%%%%%%%%%%%%%%%%%%%%%%%%%%%%%%%%%%%%%%%%%%%%%%%%%%%%%%%%
%                                                                              %
%                            D O C U M E N T   C A P                           %
%                                                                              %
%%%%%%%%%%%%%%%%%%%%%%%%%%%%%%%%%%%%%%%%%%%%%%%%%%%%%%%%%%%%%%%%%%%%%%%%%%%%%%%%
%%%%%%%%%%%%%%%%%%%%%%%%%%%%%%%%%%%%%%%%%%%%%%%%%%%%%%%%%%%%%%%%%%%%%%%%%%%%%%%%
\documentclass[epsfig,12pt]{article}
\usepackage{epsfig}
\usepackage{graphicx}
\usepackage{rotating}
\usepackage{latexsym}
\usepackage{amsmath}
\usepackage{amssymb}
\usepackage{relsize}
\usepackage[top=0.5in, right=0.5in, left=0.5in, bottom=0.8in]{geometry}
\geometry{letterpaper}
\usepackage{color}
\usepackage[usenames,dvipsnames,svgnames,table]{xcolor}
\usepackage{bm}
\usepackage{slashed}
\usepackage{enumitem}
\usepackage{tabularx}
\usepackage{booktabs}
\usepackage{multirow}
\usepackage{wrapfig}
\usepackage{fixmath}
\usepackage{esint}
\usepackage{verbatim}
\usepackage{wesa}
\usepackage{aurical}
\usepackage{pifont}
\usepackage[colorlinks]{hyperref}
\usepackage{array}
\usepackage{pgfornament}
\usepackage{siunitx}
\usepackage[europeanresistors]{circuitikz}
\usetikzlibrary{decorations.pathmorphing,patterns,arrows.meta,calc,decorations.markings}
\usetikzlibrary{math,shadows,3d,shapes,decorations,intersections}




%%%%%%%%%%%%%%%%%%%%%%%%%%%%%%%%%%%%%%%%%%%%%%%%%%%%%%%%%%%%%%%%%%%%%%%%%%%%%%%%
%%%%%%%%%%%%%%%%%%%%%%%%%%%%%%%%%%%%%%%%%%%%%%%%%%%%%%%%%%%%%%%%%%%%%%%%%%%%%%%%
%                                                                              %
%                       D O C U M E N T   S E T T I N G S                      %
%                                                                              %
%%%%%%%%%%%%%%%%%%%%%%%%%%%%%%%%%%%%%%%%%%%%%%%%%%%%%%%%%%%%%%%%%%%%%%%%%%%%%%%%
%%%%%%%%%%%%%%%%%%%%%%%%%%%%%%%%%%%%%%%%%%%%%%%%%%%%%%%%%%%%%%%%%%%%%%%%%%%%%%%%
\def\baselinestretch{1.1}
% Make the bullet list symbol a small dot
\renewcommand{\labelitemi}{{\Large$\cdot$}}
%
% This setting adjusts the units to be displayed with a central dot symbol for multiplication, e.g.
%     \si{20}{\kg\metre\per\square\second} 
% will display as
%     20 kg·m/s²
\sisetup{per-mode = symbol,
	inter-unit-product = \ensuremath{{}\cdot{}}}



%%%%%%%%%%%%%%%%%%%%%%%%%%%%%%%%%%%%%%%%%%%%%%%%%%%%%%%%%%%%%%%%%%%%%%%%%%%%%%%%
%%%%%%%%%%%%%%%%%%%%%%%%%%%%%%%%%%%%%%%%%%%%%%%%%%%%%%%%%%%%%%%%%%%%%%%%%%%%%%%%
%                                                                              %
%                                D O C U M E N T                               %
%                                                                              %
%%%%%%%%%%%%%%%%%%%%%%%%%%%%%%%%%%%%%%%%%%%%%%%%%%%%%%%%%%%%%%%%%%%%%%%%%%%%%%%%
%%%%%%%%%%%%%%%%%%%%%%%%%%%%%%%%%%%%%%%%%%%%%%%%%%%%%%%%%%%%%%%%%%%%%%%%%%%%%%%%
\begin{document}


%%%%%%%%%%%%%%%%%%%%%%%%%%%%%%%%%%%%%%%%%%%%%%%%%%%%%%%%%%%%%%%%%%%%%%%%%%%%%%%%
%                                                                              %
%                                                                              %
%                                   T I T L E                                  %
%                                                                              %
%                                                                              %
%%%%%%%%%%%%%%%%%%%%%%%%%%%%%%%%%%%%%%%%%%%%%%%%%%%%%%%%%%%%%%%%%%%%%%%%%%%%%%%%
\begin{center}
	\textbf{ \huge \color{RoyalBlue} Report for Laboratory \underline{\hskip 1cm}  }\\[2mm]
	\textbf{\textit{ \Large \color{VioletRed} Title of the Laboratory  }}\\[2mm]
	\textit{ \large \color{BlueViolet} Your Course Name  }
\end{center}
\hrule
\bigskip
\begin{tabularx}{\textwidth}{lXr}
%
	&
	\textbf{Name:}
	&
	\textbf{Laboratory Section:}
	\hspace{1.0cm}
	\\[2mm]
%
	&
	\textbf{Partners:}
\end{tabularx}
\medskip


%%%%%%%%%%%%%%%%%%%%%%%%%%%%%%%%%%%%%%%%%%%%%%%%%%%%%%%%%%%%%%%%%%%%%%%%%%%%%%%%
%                                                                              %
%                                 P U R P O S E                                %
%                                                                              %
%%%%%%%%%%%%%%%%%%%%%%%%%%%%%%%%%%%%%%%%%%%%%%%%%%%%%%%%%%%%%%%%%%%%%%%%%%%%%%%%
\section*{\textit{Purpose}}

	The purpose in no more than one sentence --- typically a laboratory tests a law or a principle,
	usually in the form of an equation (for example $ F ~=~ m a $).
	More rarely the goal of the laboratory is the acquaintance with 
	certain concepts (such as energy, momentum, force, voltage \emph{etc})


%%%%%%%%%%%%%%%%%%%%%%%%%%%%%%%%%%%%%%%%%%%%%%%%%%%%%%%%%%%%%%%%%%%%%%%%%%%%%%%%
%                                                                              %
%                               P R O C E D U R E                              %
%                                                                              %
%%%%%%%%%%%%%%%%%%%%%%%%%%%%%%%%%%%%%%%%%%%%%%%%%%%%%%%%%%%%%%%%%%%%%%%%%%%%%%%%
\section*{\textit{Overview}}

	The most difficult part --- the core of understanding the laboratory.
	In at most four sentences show how the goal is achieved.
	For example, if the laboratory tests the law $ F ~=~ m a $,
	you should briefly describe how you measure (or calculate) the force,
	and how you measure the acceleration.
	That is, you are independently experimentally measuring the left-- and right--hand sides 
	of the equation, and then checking that they agree with each other.
	The \emph{results} are not discussed here.
	Use present tense for your discussion
	

%%%%%%%%%%%%%%%%%%%%%%%%%%%%%%%%%%%%%%%%%%%%%%%%%%%%%%%%%%%%%%%%%%%%%%%%%%%%%%%%
%                                                                              %
%                            O B S E R V A T I O N S                           %
%                                                                              %
%%%%%%%%%%%%%%%%%%%%%%%%%%%%%%%%%%%%%%%%%%%%%%%%%%%%%%%%%%%%%%%%%%%%%%%%%%%%%%%%
\section*{\textit{Observations}}

	Here go the measurements.
	Data is most efficiently represented in the form of tables.
	However, before (or sometimes after) each table you need to explain what you were measuring.
	In doing so, it is also important to reference the data correctly.
	For instance, if the cart is the only moving object in the experiment, you could think
	that it would be obvious that $ v $ is the speed of the cart, and not of the sleigh of Santa Claus.
	However, scientifically, you cannot make an assumption that $ v $ would necessarily mean the speed
\begin{itemize}
\item
	each \emph{quantity} must come explained --- you cannot rely on the laboratory hand--out in assuming
	that it is obvious that $ v $ is the speed of the cart

\item
	it must be clear how each quantity was measured or obtained

\item
	each \emph{quantity} must be assigned a \emph{unique} letter --- $ v $, $ a $, $ m $, $ M $, $ R $ and so on

\item
	note here that, repetitive measurements of the same quantity do not need different letters

\item
	at the same time, if you have velocities or masses of two or more objects, then you can differentiate
	them by putting subscripts $ v_1 $, $ v_2 $, $ v_\text{avg} $, $ v_\text{in} $, $ v_\text{fin} $ or capitalizing one of them

\item
	each quoted number, besides being assigned a letter, must be accompanied by the appropriate units

\item
	if there is a group of similar measurements, it is recommended to specify the units
	only once per group, like in the table below

\item
	only mathematical and physical constants ($ \pi $, $ e $, $ g $ and so on) do not need explanation, unless a confusion can be made

\item
	avoid excess number of data tables --- for instance, if a large number of repetitive or similar measurements are taken,
	of which you will only need the average values,
	provide summarizing tables only, if that is enough for the following analysis and conclusions

\item
	clarity and conciseness are key --- as everywhere else in the report
\end{itemize}

	Note, again, that any algebraic \emph{letter} must come defined, \emph{i.e.} do not assume that $ F $ is force or $ P $ is pressure,
	unless you specifically say so.
	While deep technical details are not necessary, it must still be clear how each number was measured.
	
	Example of a data table (which must be preceded by explanations),
\begin{center}
\begin{tabular}{ccc}
%
\toprule
%
	Trial	&	$ F $ (\si{\newton})	&	$ a $ (\si{\metre\per\square\second})	\\[2mm]
%
\midrule
%
	1	&	7.88	&	3.80	\\[2mm]
	2	&	7.95	&	4.08	\\[2mm]
	3	&	8.20	&	4.14	\\[2mm]
	4	&	8.12	&	3.92	\\[2mm]
%
\bottomrule
\end{tabular}
\end{center}
	For the above measurements, we used the hanging mass $ m ~=~ \SI{2.0}{\kg} $.
	The heavier mass of the counter--weight was $ M ~=~ 5.0 \pm \SI{0.01}{\kg} $.


%%%%%%%%%%%%%%%%%%%%%%%%%%%%%%%%%%%%%%%%%%%%%%%%%%%%%%%%%%%%%%%%%%%%%%%%%%%%%%%%
%                                                                              %
%                            C A L C U L A T I O N S                           %
%                                                                              %
%%%%%%%%%%%%%%%%%%%%%%%%%%%%%%%%%%%%%%%%%%%%%%%%%%%%%%%%%%%%%%%%%%%%%%%%%%%%%%%%
\section*{\textit{Calculations}}

	This is the \textit{Calculations} or \textit{Analysis} section.
	Short calculations can go into the \textit{Observations} section, if this makes sense.
	For instance, the calculation of the average speed or mass can be quoted
	in the \textit{Observations} section.
	In some cases, the \textit{Observations} section can be merged into the \textit{Calculations} section
	entirely --- if this makes more sense for you.

	You \emph{do not need} to explain how to calculate the average.
	However, more involved calculations need an explanation.
	A formula must be quoted whenever applicable.
	If such a formula is a physics law, such as the Newton's Law, or momentum conservation law ---
	it should be quoted as the origin of the formula.

	The same rules for naming the quantities as in \emph{Observations} apply here.
	In fact, in this section you will more likely introduce additional quantities.
	Therefore, it is important to be systematic and careful with naming them,
	referring to them, and quoting the numerical values with the appropriate units.

	We calculate the energy of Millennium Falcon according to the Einstein--Palpatine formula,
\[
	E  ~~=~~  \gamma\, m c^2,
\]
	where $ m $ is the mass of the spacecraft, $ c $ is the speed of light
	and $ \gamma $ is the relativistic factor.
	Using the values $ m ~=~ \SI{45000}{\kg} $, $ c ~=~ \SI{3.0d8}{\metre\per\second} $
	and $ \gamma ~=~ 1.4 $ we find
\[
 	E ~=~ \SI{5.7d21}{\joule}.
\]
	We expect therefore, the gravitational warp drive to be the most effective
	when the spacecraft has already exited into the hyperspace.

	Repetitive calculations must be gathered into a result table,
\begin{center}
\begin{tabular}{lcc}
%
\toprule
%
	Spacecraft			&	$ m $ ($\times 10^3\,\si{\kg}$)	&	$ R $ (kiloparsec)	\\[2mm]
%
\midrule
%
	\textsl{Millennium Falcon}	&	45	&	1.6	\\[2mm]
	\textsl{Radiant VII}		&	620	&	2.1	\\[2mm]
	\textsl{Rogue Shadow}		&	58	&	1.7	\\[2mm]
	\textsl{CR90 Corvette}		&	440	&	2.4	\\[2mm]
%
\bottomrule
\end{tabular}
\end{center}
	Here $ m $ is the gross relativistic mass of the spacecraft, and $ R $ is the optimal shot range.

	Error analysis also belongs here.
	However, if this analysis is quite short, it can be moved to \emph{Conclusion} section
	if this makes more sense.


%%%%%%%%%%%%%%%%%%%%%%%%%%%%%%%%%%%%%%%%%%%%%%%%%%%%%%%%%%%%%%%%%%%%%%%%%%%%%%%%
%                                                                              %
%                              C O N C L U S I O N                             %
%                                                                              %
%%%%%%%%%%%%%%%%%%%%%%%%%%%%%%%%%%%%%%%%%%%%%%%%%%%%%%%%%%%%%%%%%%%%%%%%%%%%%%%%
\section*{\textit{Conclusion}}

	Up to six sentences discussing the results.
	Have you succeeded to test the law that you claimed in \emph{Procedure} section?
	In essence, you are answering whether the left--hand side came out equal to the right--hand side of the equation,
	but in physical terms.
	If the laboratory involved learning physical or mathematical concepts, you can comment on whether
	you were successful in that.
	Brief, summarizing data table(s) need to be given, if applicable, even if they have already been
	presented in the \textit{Calculations} section.

	If the measurements or the outcome of the laboratory did not lead to the agreement of the experiment
	with a physics law, you need to address it, by discussing possible sources of error.
	Do not spend more than two sentences for a given equation or law.

	We have measured the average in--flight speeds of the fairies, and can compare the predicted and observed
	amounts of work they perform during the day in their natural habitat
\begin{center}
\begin{tabular}{lccc}
%
\toprule
%
	Fairy		&	$ v $ (\si{in/\second})	&	$ W_\text{theory} $ ({\color{Cerulean}\ding{100}}/\si{\second}) 
							&	$ W_\text{observed} $ ({\color{Cerulean}\ding{100}}/\si{\second}) \\[2mm]
%
\midrule
%
	\textsl{Tinker Bell}	&	18.60			&	7.92			&	8.02	\\[2mm]
	\textsl{Periwinkle}	&	18.22			&	8.05			&	4.32	\\[2mm]
	\textsl{Silvermist}	&	17.56			&	6.73			&	7.14	\\[2mm]
	\textsl{Iridessa}	&	17.92			&	7.30			&	7.56	\\[2mm]
%
\bottomrule
\end{tabular}
\end{center}
	We observe a perfect agreement of our results with the equation
\[
	W_\text{fairy}	~~=~~	\text{\color{DarkOrchid}\Large\ding{96}} \times \frac{1}{2}\,m v^2.
\]
	Since Periwinkle is a winter fairy, the agreement is not as well pronounced for her,
	because naturally we had less statistics available.
	However, because she is Tinker Bell's twin sister, we expect that her performance should be identical
	and therefore the equation still to be correct.
	

\newpage
%%%%%%%%%%%%%%%%%%%%%%%%%%%%%%%%%%%%%%%%%%%%%%%%%%%%%%%%%%%%%%%%%%%%%%%%%%%%%%%%
%                                                                              %
%                   L A B O R A T O R Y   E V A L U A T I O N                  %
%                                                                              %
%%%%%%%%%%%%%%%%%%%%%%%%%%%%%%%%%%%%%%%%%%%%%%%%%%%%%%%%%%%%%%%%%%%%%%%%%%%%%%%%
\section*{\textit{Laboratory evaluation}}

	Please provide feedback on the following areas, comparing this laboratory to your previous labs.
	Please assign each of the listed categories a value in 1 ~--~ 5, with 5 being the best, 1 the worst.
	The easiest is to emphasize your assessments in the table below with bold font ---
	such as \textbf{1}, \textbf{2}, \textbf{3}, \textbf{4} or \textbf{5}

\begin{itemize}
\item
	how much fun you had completing this laboratory?
	\hfill 1\qquad 2\qquad 3\qquad 4\qquad 5

\item
	how well the lab preparation period explained this laboratory?
	\hfill 1\qquad 2\qquad 3\qquad 4\qquad 5

\item
	the amount of work required compared to the time allotted?
	\hfill 1\qquad 2\qquad 3\qquad 4\qquad 5

\item
	your understanding of this laboratory?
	\hfill 1\qquad 2\qquad 3\qquad 4\qquad 5

\item
	the difficulty of this laboratory?
	\hfill 1\qquad 2\qquad 3\qquad 4\qquad 5

\item
	how well this laboratory tied in with the lecture?
	\hfill 1\qquad 2\qquad 3\qquad 4\qquad 5
\end{itemize}

	Comments supporting or elaborating on your assessment can also be very helpful in improving the future laboratories
	--- replace this text with your comments


\end{document}
